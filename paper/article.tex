\documentclass[10pt]{article}
\pdfoutput=1 % For arXiv

\usepackage[preprint]{tmlr}
\usepackage{rawfonts, amsmath, amsfonts, graphicx, caption, subfig, makecell, lastpage, algorithm, algpseudocode}
\usepackage[hidelinks]{hyperref}
\usepackage{url}

\setcounter{totalnumber}{2}
\setcounter{topnumber}{1}

\DeclareUnicodeCharacter{2212}{-}
\newcommand{\fix}{\marginpar{FIX}}
\newcommand{\new}{\marginpar{NEW}}


\begin{document}
\title{Title goes here}


\author{\name Ola Nordmann \email ola@ntnu.no\\ \addr Norwegian University of Science and Technology,\\ Department of Computer Science,\\ NO-7491 Trondheim, Norway}


\maketitle


\begin{abstract}
  An abstract should be short but include: What is your paper about? Why is it important? How did you do it? What did you find? Why are your findings important?

  You can add a link to the corresponding software repository, if any, by using a footnote. For example: The results\footnote{The source code to reproduce the results can be found at https://gitlab.com/ola/project} show...
\end{abstract}

\section{Introduction}

Citation examples:

In \citet{krizhevsky2012imagenet}, a deep convolutional neural network was used to achieve state-of-the-art results in the ImageNet Large Scale Visual Recognition Challenge 2010.

Machine learning models trained through backpropagation have become widely popular in the last decade
since AlexNet \citep{krizhevsky2012imagenet}.

\section{Related Work}

Related previous work performed by other researchers should be presented here.

\section{Methods}

The reader, which has a background in machine learning, should be able to reproduce your results based on this section.

\section{Results and Discussion}

\section{Conclusion and Future Work}

\bibliography{article}
\bibliographystyle{tmlr}
\end{document}
